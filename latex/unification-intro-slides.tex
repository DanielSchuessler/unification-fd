% ***N.B. this uses PSTricks and so you need to use latex not pdflatex***
\documentclass[pdf,fyma2]{prosper} % Slide show (with "fyma2" style)
% The local fyma2 corrects some (new) alignment errors with the
% left side of the background gradation, the top and bottom lines,
% and the slide titles.
%
% Apparently prosper went obsolete when I wasn't looking. Powerdot
% looks like the non-obsolete successor to ha-prosper which is an
% (also obsolete) successor to prosper. Powerdot has a version of
% fyma, but it has the same alignment problem...
%
% fymafaintblue {0.88 0.95 1.00}
% fymalightblue {0.43 0.61 0.84} The bullets and lines
% fymablue      {0.24 0.45 0.70} The \ColorFoot (only marginally darker)
% fymadarkblue  {0.14 0.34 0.55} The regular text
% fymaroyalblue {0.06 0.25 0.41} The title text
\newgray{gris1}{.30}           % This was gris3 in rico/ricoDeux
\newgray{gris2}{.85}
\newrgbcolor{ocean}{0.00392156863 0.6 0.592156863}        % 0x01,99,97
\newrgbcolor{wine}{0.321568627 0.00392156863 0.333333333} % 0x52,01,55
% TODO: how to use \expandafter in order to redefine \emph?
\newcommand{\unhilight}[1]{{\color{gris2}#1}}
\newcommand{\hilighten}[1]{{\color{fymablue}#1}}
%% ~~~~~~~~~~~~~~~~~~~~~~~~~~~~~~~~~~~~~~~~~~~~~~~~~~~~~~~~~~~~~

%\hypersetup{pdfpagemode=FullScreen} % make AcrobatReader auto full-screen

\title{A Traditional Introduction to Unification}
%\subtitle{}
\date{7 July 2011}
\author{wren ng thornton}
\email{wrnthorn@indiana.edu}
\institution{%
    Cognitive Science \& Computational Linguistics\\
    Indiana University Bloomington}
% N.B. entitlement.sty breaks prosper somehow...
\slideCaption{wren ng thornton \quad A Traditional Introduction to Unification}


% BUGFIX: this line fixes the erroneous "Error: Math version `bold'
% is not defined." error caused by packaging problems
\DeclareMathVersion{bold}
% BUG: \llparenthesis\rrparenthesis has no displaystyle
\usepackage{stmaryrd}

\usepackage{amsmath,amssymb,latexsym}% defines \text{} in math mode
% BUG: this makes the math much prettier, but it also adds serifs to other things
\usepackage{mathptmx} % mathtimes

\usepackage{pstricks,pst-node}

\usepackage{listings}
\lstloadlanguages{Haskell,Prolog}

\newcommand{\opr}[1]{\ensuremath{\operatorname{#1}}}
\newcommand{\var}[1]{\ensuremath{\text{\textrm{\textit{#1}}}}}
\newcommand{\cat}[1]{\ensuremath{\mathbf{#1}}}

\newcommand{\NAT}{\ensuremath{\mathbb{N}}}
\newcommand{\WHOLE}{\ensuremath{\mathbb{N}_{{>}0}}}
\newcommand{\SYM}[1][]{\ensuremath{\var{Sym}_{#1}}}
\newcommand{\ARITY}[1][]{\ensuremath{\var{Arity}_{#1}}}
\newcommand{\TERMS}[2][]{\ensuremath{\mathbb{T}_{#2}^{#1}}}
\newcommand{\FREEVARS}{\ensuremath{\var{FV}}}
\newcommand{\UNIFIERS}{\ensuremath{\mathcal{U}\!}}

\newcommand{\unify}{\ensuremath{\doteq}}
\newcommand{\isom}{\ensuremath{\cong}}
\newcommand{\notisom}{\ensuremath{\!\not{\!\!\isom\,}\,}}
\newcommand{\subsume}{\ensuremath{\lessdot}}
\newcommand{\subsumeq}{\ensuremath{\,\,\underline{\!\subsume\!}\,\,}}
\newcommand{\notsubsume}{\ensuremath{\!\not\!\!\subsume\,}}

% N.B., \it and \bf use serif fonts; whereas \textit, \textbf use sans-serif
\renewcommand{\emph}[1]{\hilighten{\textbf{#1}}}
\newcommand{\defn}[1]{\textbf{#1}}

\makeatletter
\newcommand{\thm}[1][]{%
    \def\thm@tmp{#1}%
    \textbf{Theorem\ifx\thm@tmp\@empty\relax\else\ (\thm@tmp)\fi:}\ }
\makeatother

%% ~~~~~~~~~~~~~~~~~~~~~~~~~~~~~~~~~~~~~~~~~~~~~~~~~~~~~~~~~~~~~
%% ~~~~~~~~~~~~~~~~~~~~~~~~~~~~~~~~~~~~~~~~~~~~~~~~~~~~~~~~~~~~~
\begin{document}

\maketitle

%% ~~~~~~~~~~~~~~~~~~~~~~~~~~~~~~~~~~~~~~~~~~~~~~~~~~~~~~~~~~~~~
\begin{slide}{Outline of the talk}
	\begin{itemize}
	\item Why ``traditional''?
	\item Preliminary Definitions
	    \begin{itemize}
    	\item \hilighten{Ranked alphabets, Variables}
    	\item \hilighten{Herbrand universes, Terms}
    	\item \hilighten{Bindings, Substitutions}
    	\item \hilighten{Sorts}
    	\end{itemize}
	\item Unification
	    \begin{itemize}
    	\item \hilighten{Subsumption}
    	\item \hilighten{Unifiers, CMU Sets, MGUs}
    	\item \hilighten{Some theorems}
    	\end{itemize}
	\item Uses of Unification
	    \begin{itemize}
    	\item \hilighten{Case analysis}
    	\item \hilighten{Logic programming}
    	\item \hilighten{Type checking, inference}
    	\item \hilighten{Unification-based parsing}
    	\end{itemize}
	\end{itemize}
\end{slide}

%% ~~~~~~~~~~~~~~~~~~~~~~~~~~~~~~~~~~~~~~~~~~~~~~~~~~~~~~~~~~~~~
\begin{slide}{Why ``traditional''?}
	\vspace{1em}
	\begin{itemize}
	\item Well, why \emph{not} traditional?
        \begin{itemize}
	    \item The preliminary definitions are overly complicated (and overly specific) due to a set-theoretic heritage
        \end{itemize}
	\vspace{1em}
	\item They can be generalized and clarified by a categorical treatment
        \begin{itemize}
        \item E.g., Goguen (1989)
        \item But I don't think I've taught you enough category theory for that to be worth pursuing here
        \end{itemize}
	\vspace{1em}
    \item The set-theoretic terminology is worth being familiar with
        \begin{itemize}
        \item It's still in wide use
        \item It's essential for understanding any of the ``old'' papers
            \begin{itemize}
            \item i.e., late-1980s through mid-1990s
            \item let alone papers which actually are old (e.g.\ 1960s)
            \end{itemize}
        \end{itemize}
	\end{itemize}
\end{slide}


\begin{slide}{Ranked Alphabets \& Variables}
	\vspace{1em}
	\begin{itemize}
	\item Let \NAT\ denote the set $\{0,1,2,\ldots\}$ of natural numbers
	    \begin{itemize}
	    \item Let \WHOLE\ denote the set $\{1,2,\ldots\}$ of whole numbers
	    \end{itemize}
	\vspace{1em}
	\item A (singly) \defn{ranked alphabet} $\mathcal{F}$ is a finite set of abstract elements equipped with two projection functions
	    \begin{itemize}
	    \item $\SYM[\mathcal{F}]$ taking $\mathcal{F}$ to function symbols called \defn{constructors}$\null^\dagger$
	        \begin{itemize}
	        \item surjective, by definition
	        \item injective for the singly-ranked case
	        \end{itemize}
	    \item $\ARITY[\mathcal{F}] : \mathcal{F} \to \NAT$
	    \end{itemize}
    \item Let $\mathcal{F}_p$ denote the subset $\{ f : \mathcal{F} \mid \ARITY(f) = p \}$
    \item The subset $\mathcal{F}_0$ are called \defn{constants}
	\vspace{1.5em}
	\item Let $\mathcal{X}$ be an RA containing only constants, called \defn{variables}
        \begin{itemize}
        \item Without loss of generality, we assume $\mathcal{X}$ and $\mathcal{F}_0$ are disjoint
        \end{itemize}
	\end{itemize}
\end{slide}

\begin{slide}{Herbrand universes \& Terms}
	\vspace{1em}
	\begin{itemize}
    \item A \defn{Herbrand universe} $\TERMS{A}$ is a set of \defn{terms} over an RA $A$
    \item Namely, the smallest set such that
\[
\begin{array}{rl}
 & (\forall f\in A_0.~
	~\SYM(f) \in \TERMS{A}) \\
\land & (\forall p \in \WHOLE.~
	\forall f\in A_p.~
		\forall t_1,\ldots,t_p \in \TERMS{A}.~
			~\SYM(f)(t_1,\ldots,t_p) \in \TERMS{A}) \\
\end{array}
\]
    \item Let $\FREEVARS : \TERMS{\mathcal{F}\uplus\mathcal{X}} \to 2^{\mathcal{X}}$ map terms to a set of \defn{free variables}
    	\begin{itemize}
    	\item $\forall t \in \TERMS{\mathcal{F}\uplus\mathcal{X}}.~~ t \in \TERMS{\mathcal{F}\uplus\FREEVARS(t)}$
    	\end{itemize}
    \item A term $t \in \TERMS{\mathcal{F}\uplus\mathcal{X}}$ is called \defn{ground} or \defn{closed}$\null^\dagger$ iff $\FREEVARS(t) = \emptyset$
    \item A term is called \defn{linear} if each free variable occurs at most once
	\vspace{1em}
    \item Since $\mathcal{X}$ is defined essentially by its cardinality
        \begin{itemize}
        \item Without loss of generality, we will consider $\TERMS{\mathcal{F}\uplus\mathcal{X}}$ an $\alpha$-equivelence class over all $\TERMS{\mathcal{F}\uplus\mathcal{X'}} \opr{~s.t.~} \lvert\mathcal{X'}\rvert = \lvert\mathcal{X}\rvert$
        \end{itemize}
	\end{itemize}
\end{slide}

\begin{slide}{Here be Dragons}
	\begin{itemize}
    \item A \defn{binding} is a mapping from variables to terms
        \begin{itemize}
        \item Issue: ``unbound'' variables?
        \item Issue: idempotence?
        \item Let us call it a total function $s : \mathcal{X'} \to \TERMS{\mathcal{F}\uplus\mathcal{X''}}$ where $\mathcal{X'}\cap\mathcal{X''}=\emptyset$ and $\mathcal{X''} = \bigcup_{x\in \mathcal{X'}} \FREEVARS(s(x))$
        \end{itemize}
    \item A \defn{substitution} $\sigma_s : \TERMS{\mathcal{F}\uplus\mathcal{X}} \to \TERMS{\mathcal{F}\uplus\mathcal{X}}$ expresses the recursive application of bindings $s : \mathcal{X'} \to \TERMS{\mathcal{F}\uplus(\mathcal{X}\setminus\mathcal{X'})}$
        \begin{itemize}
        \item $\sigma_s = {\llparenthesis s + \opr{id}_{\mathcal{X}\setminus\mathcal{X'}} + \opr{id}_\mathcal{F} \rrparenthesis}_{\mathcal{F}\uplus\mathcal{X}}$
        \item Substitutions are idempotent, by definition
        \item We don't care about the details of bindings, so long as substitutions are correct
        \end{itemize}
    \item Substitutions form a category
        \begin{itemize}
        \item Objects are term universes
        \item There are identity substitutions: $\sigma_\emptyset = \opr{id}_{\TERMS{\mathcal{F}\uplus\mathcal{X}}}$
        \item Substitutions can be composed: $(\sigma \circ \theta) t = \sigma (\theta t)$
        \end{itemize}
	\end{itemize}
\end{slide}

\begin{slide}{Sorts}
	\begin{itemize}
    \item So far we've been talking about ``untyped'' terms
    \item A \defn{many-sorted ranked alphabet} is an RA with an additional projection function $\Sigma_\mathcal{F} : \mathcal{F} \to \Sigma$ where
    	\begin{itemize}
        \item $\Sigma$ is some non-empty set of \defn{sorts}
        \item $\Sigma$ is equipped with $(\prec)$: subsort relation, a partial ordering
        \item $\Sigma$ is equipped with $(\curlywedge)$: infimum sort relation
        \item $\Sigma_\mathcal{F}^0(f)$ returns the codomain sort for $f$
        \item $\forall i : 0 < i \leq \ARITY(f).~ \Sigma_\mathcal{F}^i(f)$ returns the $i$th domain sort for $f$
        \end{itemize}
    \item \defn{Well-sorted terms} ($\Sigma$-terms) can be defined in the obvious way
    	\begin{itemize}
        \item When $\Sigma$ is the singleton set (as in single-sorted RAs),
            \\ $\Sigma$-terms are the $\Sigma_\mathcal{F}$-word algebra on $\mathcal{X}$
            \\ (aka the free monad generated by $\mathcal{F}$)
        \item When $\Sigma$ is ordered by equality,
            \\ $\Sigma$-terms are the free $\Sigma_\mathcal{F}$-algebra generated by $\mathcal{X}$
        \end{itemize}
	\end{itemize}
\end{slide}

%% ~~~~~~~~~~~~~~~~~~~~~~~~~~~~~~~~~~~~~~~~~~~~~~~~~~~~~~~~~~~~~
\begin{slide}{Outline of the talk}
	\begin{itemize}
	\item \unhilight{Why ``traditional''?}
	\item \unhilight{Preliminary Definitions}
	    \begin{itemize}
    	\item \unhilight{Ranked alphabets, Variables}
    	\item \unhilight{Herbrand universes, Terms}
    	\item \unhilight{Bindings, Substitutions}
    	\item \unhilight{Sorts}
    	\end{itemize}
	\item Unification
	    \begin{itemize}
    	\item Subsumption
    	\item Unifiers, CMU Sets, MGUs
    	\item Some theorems
    	\end{itemize}
	\item \unhilight{Uses of Unification}
	    \begin{itemize}
    	\item \unhilight{Case analysis}
    	\item \unhilight{Logic programming}
    	\item \unhilight{Type checking, inference}
    	\item \unhilight{Unification-based parsing}
    	\end{itemize}
	\end{itemize}
\end{slide}

\begin{slide}{Subsumption}
	\begin{itemize}
	\item Let $(\isom)$ denote an equivalence relation on terms
    	\begin{itemize}
    	\item For now, equality is sufficient
    	\end{itemize}
	\item For terms $s,t$ if there exists a substitution $\sigma \opr{~s.t.~} \sigma s \isom t$
    	\begin{itemize}
	    \item We say that $s$ \defn{generalizes} or \defn{subsumes} $t$
	    \item Dually, we say that $t$ \defn{refines} or \defn{specializes} $s$
	    \item We denote the existence of $\sigma$ by saying $s \subsumeq t$
    	\item If the substitution is non-empty, then we say $s \subsume t$
    	\end{itemize}
    \item Subsumption can be lifted to substitutions too
    	\begin{itemize}
    	\item We say $\sigma \subsumeq \theta$ iff $\exists \rho.~ \rho\circ\sigma \isom \theta$
    	\item We say $\sigma \subsume \theta$ iff $\sigma \subsumeq \theta \land \sigma \notisom \theta$
    	\end{itemize}
    \item For a set $S$ of substitutions, $\sigma\in S$ is called a \defn{least substitution} iff $\forall \theta\in S.~ \sigma \subsumeq \theta$
    	\begin{itemize}
    	\item N.B., this means $\sigma\in S \opr{~s.t.~} \forall \theta\in S.~ \exists \rho.~~ \rho\circ\sigma \isom \theta$
            \begin{itemize}
            \item i.e., $\rho$ need not be in $S$
            \item and $\rho$ can be different for each $\theta$
            \end{itemize}
    	\end{itemize}
	\end{itemize}
\end{slide}

\begin{slide}{Unification}
	\vspace{1em}
	\begin{itemize}
	\item \defn{Unification} is an associative commutative operation on terms
        \begin{itemize}
        \item Equivalent to consider it an operation on (finite) sets of terms
        \end{itemize}
    \item A \defn{unifier} of a set $D$ of terms is a substitution
        \\ $\sigma \opr{~s.t.~} \forall d,d'\in D.~ \sigma d \isom \sigma d'$
    \item Let $\UNIFIERS(D)$ denote the set of all unifiers of $D$
    \item $D$ is called \defn{unifiable} if $\UNIFIERS(D) \neq \emptyset$
        \begin{itemize}
        \item We say $s \doteq t$ is true when $\{s,t\}$ is unifiable
        \end{itemize}
    \item A subset $U \subset \UNIFIERS(D)$ is \defn{complete} iff $\forall \sigma \in \UNIFIERS(D).~ \exists \theta \in U.~ \theta \subsumeq \sigma$
    \item A subset $U \subset \UNIFIERS(D)$ is \defn{minimal} iff $\forall \sigma,\theta \in U.~ \sigma \notsubsume \theta$
    \item A \defn{cmu set} is a complete and minimal set of unifiers
    \item If a cmu set is singleton, we call its element a \defn{most general unifier} (mgu)
        \begin{itemize}
        \item When there is only one cmu set, so we call it \emph{the} mgu
        \end{itemize}
    \end{itemize}
\end{slide}

\begin{slide}{Unification}
	\vspace{1em}
	\begin{itemize}
	\item \thm[Robinson, 1965] for the single-sorted case, if $D$ is unifiable then there is a unique least substitution in $\UNIFIERS(D)$
	\vspace{1em}
	\item \thm[Walther, 1988] for the many-sorted case, each $\Sigma$-unifiable set of terms has a well-sorted mgu \emph{which does not introduce new variables} iff the subsort hierarchy has a forest structure
        \begin{itemize}
        \item i.e., $\forall s, s_1, s_2 \in \Sigma.~ (s \prec s_1 \land s\prec s_2) \Rightarrow (s_1\prec s_2 \lor s_2\prec s_1)$
        \end{itemize}
	\vspace{1em}
	\item \thm[Walther, 1988] if the subsort hierarchy is a meet-semilattice, then every $\Sigma$-unifiable set of terms has a singleton cmu set (though it may introduce auxilliary variables)
    \end{itemize}
\end{slide}

\begin{slide}{Higher-Order Unification}
	\begin{itemize}
	\item So far we've been discussing first-order unification
	\item \defn{Higher-order unification} adds lambdas to the term language
    	\begin{itemize}
    	\item The possibilities for what this could mean are enormous
    	\item Often people mean Church's simply-typed $\lambda$-calculus modulo $\beta\eta$-equivalence ($\lambda_{\rightarrow}/\beta\eta$)
    	\end{itemize}
	\vspace{1em}
	\item \thm Higher-order unification is undecidable in general
	% ; though it is semi-decidable for decidable $\lambda$-calculi
    	\begin{itemize}
    	\item \textbf{Huet, 1973:} Third-order unification in $\lambda_{\rightarrow}/\beta\eta$ can be reduced to the Post correspondence problem
    	%\item \textbf{Goldfarb, 1981:} Second-order (details?) also undecidable, by reduction to Hilbert's tenth problem
    	%\item \textbf{Schubert, 1998:} Second-order undecidable by reduction to halting problem of a two-counter automaton
    	%\item \textbf{Farmer, 1991a; Levy \& Veanes, 1998:} sharpen this further based on the number of variables needed to get undecidability
    	\item \textbf{Loader, 2002:} Second-order $\lambda_{\rightarrow}/\beta$ also undecidable
    	\end{itemize}
    \item But, many interesting subsets are still decidable
    	\begin{itemize}
    	\item \textbf{Miller, 1991:} Higher-order `pattern' unification
    	\item \textbf{Schmidt-Schau\ss, 1999:} `Bounded' 2nd-order unification
    	\item \textbf{Farmer, 1988:} Unary (or `monadic') 2nd-order unification
    	\end{itemize}
    \item Many more results on both sides
    \end{itemize}
\end{slide}

%% ~~~~~~~~~~~~~~~~~~~~~~~~~~~~~~~~~~~~~~~~~~~~~~~~~~~~~~~~~~~~~
\begin{slide}{Outline of the talk}
	\begin{itemize}
	\item \unhilight{Why ``traditional''?}
	\item \unhilight{Preliminary Definitions}
	    \begin{itemize}
    	\item \unhilight{Ranked alphabets, Variables}
    	\item \unhilight{Herbrand universes, Terms}
    	\item \unhilight{Bindings, Substitutions}
    	\item \unhilight{Sorts}
    	\end{itemize}
	\item \unhilight{Unification}
	    \begin{itemize}
    	\item \unhilight{Subsumption}
    	\item \unhilight{Unifiers, CMU Sets, MGUs}
    	\item \unhilight{Some theorems}
    	\end{itemize}
	\item Uses of Unification
	    \begin{itemize}
    	\item Case analysis
    	\item Logic programming
    	\item Type checking, inference
    	\item Unification-based parsing
    	\end{itemize}
	\end{itemize}
\end{slide}

\begin{slide}{Case Analysis}
	\begin{itemize}
	\item Case analysis in functional languages is a simplified version of unification
        \begin{itemize}
        \item The scrutinee is a ground term
        \item Each pattern is a linear term
        \end{itemize}
    \item Also has higher-order versions
        \begin{itemize}
        \item 2nd-, 3rd-, and 4th-order are decidable
            \begin{itemize}
            \item 2nd- and 3rd-order are NP-complete
            \end{itemize}
        \item General case has remained an open problem for >20~years
        \end{itemize}
	\vspace{1em}
    \item Cf., clause resolution in logic programming
        \begin{itemize}
        \item \thm[Dawson et al., 1995, 1996] Determining the optimal clause resolution automata for an ordered set of clauses can be done in polynomial time; for an unordered set, it is NP-complete.
        \item $O(n^2*m*(n+m))$ for $n$ clauses with at most $m$ symbols each
        \end{itemize}
	\end{itemize}
\end{slide}

\begin{slide}{Logic Programming}
	\vspace{1em}
	\begin{itemize}
	\item You can use unification to ``run functions backwards''
\lstset{
    language=Prolog,
    basicstyle=\small\ttfamily,
    flexiblecolumns=false,
	tabsize=4,
	frameround=fttt,
    basewidth={0.5em,0.45em},
    otherkeywords={(,),\,,.,:-}, keywordstyle={\color{fymaroyalblue}},
    sensitive=true,
    keywords={[2]X,Y,Z}, keywordstyle={[2]\color{fymablue}}}
\begin{lstlisting}
z.
s(X).

plus(z, Y, Y). /* Note the non-linear pattern */
plus(s(X), Y, s(Z)) :- plus(X, Y, Z).

/* Of course we don't really need to define this */
minus(Z, X, Y) :- plus(X, Y, Z).
\end{lstlisting}
% TODO: a better example?
	\end{itemize}
\end{slide}

\begin{slide}{Exploiting Continuity}
	\vspace{1em}
	\begin{itemize}
	\item In Haskell, we can think of $\bot$ like a variable (!!)
	\vspace{1em}
	\item This lets us define an unambiguous choice operator
    	\begin{itemize}
    	\item $\var{unamb} :: \forall\alpha.~ \alpha \to \alpha \to \alpha$
    	\item Run both arguments in parallel, return the defined one
        \item \url{http://hackage.haskell.org/package/unamb}
    	\end{itemize}
    \item Which can be generalized
    	\begin{itemize}
    	\item $\opr{class}~ \var{HasLub}~\alpha ~\opr{where}~ \var{lub} :: \alpha \to \alpha \to \alpha$
    	\item Run both arguments in parallel, unify the results
        \item \url{http://hackage.haskell.org/package/lub}
    	\end{itemize}
	\vspace{1em}
	\item The same trick is exploited by lazy SmallCheck
    	\begin{itemize}
    	\item Automatic exhaustive checking for small values
    	\item \url{http://www.cs.york.ac.uk/fp/smallcheck/}
    	\end{itemize}
	\end{itemize}
\end{slide}

\begin{slide}{Type Checking \& Inference}
	\vspace{1em}
	\begin{itemize}
	\item Type checking/inference is (sometimes) surprisingly easy
	\item Walk over a term, generating a system of equations
        \begin{itemize}
        \item Give every subterm a unification variable for its type
        \item If $((f :: A) (e :: B)) :: C$ then unify $A \doteq (B \to C)$
        \item If $(\lambda x.~ e :: A) :: B$ then unify $B \doteq (B' \to A)$ for a fresh variable $B'$
            \begin{itemize}
            \item and add the local constraint $x :: B'$\ldots
            \end{itemize}
        \item etc
        \end{itemize}
    \item Solve the unification problem
    \item Generalize any remaining free unification variables
        \\ (i.e., make them rank-1 universally quantified type variables)
	\end{itemize}
\end{slide}

\begin{slide}{Type Checking \& Inference}
	\vspace{1em}
	\begin{itemize}
	\item Type theories stronger than Hindley--Milner--Damas are harder
	\vspace{1em}
        \begin{itemize}
        \item \emph{Inferring} higher-rank polymorphism isn't sound
            \begin{itemize}
            \item with enough polymorphism everything type checks
            \item though \emph{checking} higher-rank types is soluble
            \end{itemize}
	\vspace{1em}
        \item If you have type-level $\lambda$-expressions
            \begin{itemize}
            \item e.g., because of dependent types or type operators
            \item then you need higher-order unification
            \end{itemize}
        \end{itemize}
	\end{itemize}
\end{slide}

\begin{slide}{Unification-based Parsing}
	\vspace{1em}
	\begin{itemize}
	\item Some grammar formalisms for natural language parsing use unification
        \begin{itemize}
        \item HPSG, GPSG, LFG
        \end{itemize}
    \item But their unification is more complex than the single-sorted first-order unification discussed here
	\vspace{2em}
    \item Some grammar formalisms use constraint-based solvers
        \begin{itemize}
        \item XDG: Extensible Dependency Grammar
            \\ \url{http://www.ps.uni-saarland.de/~rade/xdg.html}
        \end{itemize}
    \item Constraint-based programming is subtly different from but easily allied with logic programming.
        \begin{itemize}
        \item ECLiPSe constraint-logic programming
            \\ \url{http://eclipseclp.org/}
        \end{itemize}
	\end{itemize}
\end{slide}

%% ~~~~~~~~~~~~~~~~~~~~~~~~~~~~~~~~~~~~~~~~~~~~~~~~~~~~~~~~~~~~~
%% ~~~~~~~~~~~~~~~~~~~~~~~~~~~~~~~~~~~~~~~~~~~~~~~~~~~~~~~~ fin.
% BUG: how can we use \pdfstringdef or \texorpdfstring to fix the PDF bookmark?
\part{%
\usefont{OT1}{ptm}{m}{it}\fontsize{14.4pt}{12pt}\selectfont
$\sim$fin.%
}
\end{document}
